\documentclass[a4paper,12pt,smallheadings]{scrartcl}
\usepackage[utf8]{inputenc}
\usepackage{mathtools}
\usepackage[onehalfspacing]{setspace}

%opening
\title{Business Process Change - Formulars}

\addtolength{\textheight}{2cm}
\addtolength{\voffset}{-1cm}

\makeatletter
\let\thetitle\@title
\makeatother

\begin{document}
\thetitle
\hrule

\paragraph{6 $\sigma$}

\subparagraph{Zufallsgrößen}  \hfill

$\mu$ := Mittelwert

$\sigma$ := Standardabweichung

$P\left( X \leq x\right) := P\left( X \in \left( -\infty,x\right] \right) $

\subparagraph{Normalverteilung} \hfill

$F_{\mu,\sigma}\left(x\right)   := P\left(X \leq x\right) = \frac{1}{\sqrt{2\pi}\sigma} \int_{-\infty}^{x} e^{-\frac{{\left( t-\mu\right)}^2}{2\sigma^2}} dt$

$F_{\mu,\sigma}\left(x \right) = F_{0,1}\left(\frac{x-\mu}{\sigma} \right)$

$F_{0,1}\left(-x \right) = 1-F_{0,1}\left(x \right) $

\subparagraph{Konfidenzintervalle} \hfill

$p$ := Anteil an der Gesamtheit

$n$ := Größe der Stichprobe

$S_n$ := Stichprobe

$\alpha$ := Fehlerniveau $\left(> 0 \right)$

$\delta $ := Genauigkeit der Vorhersage $\left( >0  \right)$

$\sigma = \sqrt{\frac{p\left(1-p)\right)}{n}} \leq \frac{1}{2\sqrt{n}}$

$\delta \geq \frac{1}{2\sqrt{n}}F_{0,1}^{-1}\left( 1- \frac{\alpha}{2}\right) \implies P\left( \left| S_n - p \right| \leq \delta \right) \geq 1 - \alpha $

\paragraph{Prozessanalyse}

\subparagraph{Kennzahlen} \hfill

WIP := Work-In-Process (Anzahl der Jobs in Bearbeitung (oder wartend))

$\lambda$ (Throughput) := Durchsatz

CT := Cycle Time (Zeit für die ein Job im Prozess verbringt) 

\subparagraph{Little's Law}

$WIP = \lambda * CT$

\subparagraph{Cycle Time Analysis}
\begin{align*}
&CT\left(a_1.P_1+...+a_n.P_n\right)&&= p_1\bullet\left(c\left(a_1\right)+CT\left(P_1\right)\right)+...+p_n\bullet\left(c\left(a_n\right)+CT\left(P_n\right)\right) \\
&CT\left(P_1|P_2\right) &&= max\left\lbrace CT\left(P_1\right), CT\left(P_2\right)\right\rbrace \\
&CT\left(A\right) &&= CT\left(P_A\right) \text{ for a defintion } A = P_A \\
&CT\left(\nu a P\right) &&= CT\left(P\right) \text{ with } c\left(a\right) := 0 \\
&CT\left(\epsilon\right) &&= 0
\end{align*}

\subparagraph{Capacity Analysis} \hfill

\begin{math}
\delta(a,b) = \left\lbrace
	\begin{array}{l l}
		1 & \quad \text{if } a=b \\
		0 & \quad \text{otherwise}
	\end{array} \right.
\end{math}
\begin{align*}
&\#\left(a,P\right) := \\
&\sum_{i=1}^{n} p_i\left(\delta\left(a,a\right)+\#\left(a,P_1\right)\right),& &\text{if } P=a_1.P_1+...+a_n.P_n \\
&0,& &\text{if } P=\epsilon \\
&\#\left(a,P_1\right) + \#\left(a,P_2\right),& &\text{if } P=P_1|P_2 \\
&\#\left(a,P_A\right), & &\text{if } P = A = P_A \\
&\left(1-\delta\left(a,a'\right)\right) \#\left(a,P'\right),& &\text{if } P = \nu a' P'
\end{align*}

$Capacity\left(R,P\right) := \frac{k}{\#\left(a_1,P\right)c\left(a_1\right) + ... + \#\left(a_m,P\right)c\left(a_m\right)} \quad$ R an actor with k resources

$Capacity\left(P\right):= min\left\lbrace Capacity\left(R,P\right) | \text{ R actor for P }\right\rbrace$

$Utilization \left( R,P\right) := \frac{\lambda}{Capacity \left(R,P\right)} \quad \left( \leq 1\right)$

\paragraph{Queuing Systems}

\subparagraph{Utilization factor} 
$\rho = \frac{\lambda}{c\mu}$ 
\quad (Assumption: $\frac{\lambda}{c\mu} < 1$)

$WIP = \sum_{n=0}^{K} n P_n$

$Q = \sum_{n=c}^{K} (n-c) P_n$

\subparagraph{Balance Equations} \hfill

$n > 0: \quad \lambda P_{n-1} + min\left\{n+1,c\right\}\mu P_{n+1} = \lambda P_n + min\left\{ n, c \right\} \mu P_n $

$n = 0: \quad \mu P_1 = \lambda P_0$

\hfill

\underline{Für $K \neq \infty$:}

$n = K: \quad \lambda P_{n-1} = min\left\{ n,c \right\} \mu P_{n} $

$ 1 = P_0 + P_1 + ... + P_n $  \quad für $ n \leq K$

\hfill

\underline{Für $K = \infty$:}

$WIP = \frac{\lambda}{\mu} + P_0 \frac{1}{c!} \left(\frac{\lambda}{\mu}\right)^c \frac{\rho}{\left(1-\rho\right)^2}$

$Q = WIP - \frac{\lambda}{\mu}$

$CT_Q = CT_{Total} - \frac{1}{\mu}$

$1 = P_0 + P_1 + P_2 +... \implies P_0 = \left( \sum_{n=0}^{c-1} \frac{\left(\frac{\lambda}{\mu}\right)^n}{n!} +
\frac{ \left(\frac{\lambda}{\mu}\right)^c}{c! \left(1-\rho\right)} \right)^{-1}$

$P_{n+1} = \frac{1}{\left(n+1\right)!} \left(\frac{\lambda}{\mu}\right)^{n+1}P_0$ \quad für $0 \leq n < c$

$P_n = \rho^{n-c} \frac{1}{c!} \left(\frac{\lambda}{\mu}\right)^c P_0$ \quad für $ n \geq c$

\paragraph{Prozessmodell Änderungen} 
Kontext: Ein Term der [] enthält

Sei $P \in \wp$, dann ist $S = {P' | P \rightarrow^* P'}$ die Menge aller erreichbaren Zustände (von P)

Sei $T_p = (S , \rightarrow, \downarrow)$ ein Transitionssystem für P

Sei $P \rightarrow^* P'$ über die Transitionssequenz $(P_{i-1} \overset{\alpha_i}{\rightarrow} P_i)_{i \in \{1, \dots n\}}$
\begin{itemize}
\item $(P = P_0, \alpha_1, P_1)(P_1, \alpha_2, P_2) \dots (P_{n-1}, \alpha_n, P_n = P')$ wird Instanz von P genannt (ist korrekt typisiert in P)
\item $cs(w) := P_n$ ist der aktuelle Zustand
\item $h(w): \alpha_1\alpha_2\alpha_3 \dots \alpha_n$ ist die Historie
\end{itemize}

\subparagraph{Schwach Äquivalent}

Seien P und Q Prozessausdrücke in $\wp$

Eine binäre Relation S auf $\wp$ heißt schwach äquivalent, wenn gilt:

\begin{eqnarray}
(P \overset{\tau}{\rightarrow} P') \Rightarrow (\exists Q': Q \Rightarrow Q' \wedge (P', Q') \in S) \\
(P \overset{\alpha}{\rightarrow} P') \overset{\alpha}{\Rightarrow} (\exists Q': Q \Rightarrow Q' \wedge (P', Q') \in S) \\
P \downarrow \Rightarrow (\exists Q':Q \Rightarrow Q' \wedge Q' \downarrow)
\end{eqnarray}
Wir schreiben $P \leq Q$ wenn P von Q simuliert wird.

\subparagraph{Range Region \& Ersetzung}
TODO

\subparagraph{Korrektheit}
TODO



\end{document}
